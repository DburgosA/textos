\documentclass{article}
\usepackage[utf8]{inputenc}
\usepackage[spanish]{babel}
\usepackage{longtable}
\usepackage{booktabs}
\usepackage{geometry}
\usepackage{array}
\geometry{a4paper, margin=0.5cm}
\setlength{\parindent}{0pt}
\setlength{\parskip}{0pt}
\renewcommand{\arraystretch}{1.5}
\setlength{\tabcolsep}{8pt}

\title{Cuestionario de Optimización - Preguntas y Respuestas (COMPLETO)}
\author{}
\date{}

\begin{document}

\setlength{\LTpre}{0pt}
\setlength{\LTpost}{0pt}

\maketitle

\begin{longtable}{|>{\centering\arraybackslash}p{1.5cm}|>{\raggedright\arraybackslash}p{14cm}|>{\centering\arraybackslash}p{2cm}|}
\hline
\textbf{ID} & \textbf{Pregunta} & \textbf{Respuesta} \\
\hline
\endfirsthead

\hline
\textbf{ID} & \textbf{Pregunta} & \textbf{Respuesta} \\
\hline
\endhead

\hline
\endfoot

\hline
\endlastfoot

P01 & El método de dos fases se aplica únicamente en problemas de minimización & Falso \\
\hline
P02 & En el algoritmo de recocido simulado se aceptan como soluciones posibles estados de energía más altos para tratar de salir de mínimos locales & Verdadero \\
\hline
P03 & La solución del problema de programación entera sin tomar en cuenta las restricciones de integridad se denomina relajación contínua & Verdadero \\
\hline
P04 & En un problema de ramificar y acotar es sencillo determinar la cantidad de memoria RAM que se utilizará & Falso \\
\hline
P05 & Si una función es convexa, cualquier mínimo o máximo local es también un mínimo o máximo global & Verdadero \\
\hline
P06 & Si x⋆ es un mínimo local y f(x) es continuamente diferenciable en Vx⋆, entonces ∇f(x⋆)≠0 & Falso \\
\hline
P07 & La actualización de t en el método de Barzilai-Borwein tiene como objetivo acelerar la convergencia a un punto crítico local & Verdadero \\
\hline
P08 & Un precio sombra positivo en un tablero final del método simplex implica que al hacer crecer el lado derecho de la restricción, el valor de la función objetivo siempre crecerá & Falso \\
\hline
P09 & En el algoritmo de recocido simulado la temperatura controla la duración del algoritmo & Falso \\
\hline
P10 & Una matriz es semidefinida positiva si y solo si todos sus valores propios son mayores o iguales a cero & Verdadero \\
\hline
P11 & La derivada numérica de una función es más precisa mientras más pequeño sea el valor de Δx, sin importar qué tan pequeño sea & Falso \\
\hline
P12 & Las variables de holgura se agregan cuando hay restricciones de tipo "mayor o igual que" & Falso \\
\hline
P13 & Dado que el gradiente de una función es el vector de derivadas parciales, es necesario que cada Δx sea diferente solo cuando las componentes de x son diferentes & Verdadero \\
\hline
P14 & La diferencia principal entre el algoritmo del gradiente descendente y el método de Newton es la utilización de la matriz hessiana en este último para determinar la dirección de búsqueda & Verdadero \\
\hline
P15 & Se dice que una función f(x) es convexa si para cualesquiera xa, xb y λ∈[0,1] se cumple f(λxa+(1−λ)xb)<λf(xa)+(1−λ)f(xb) & Falso \\
\hline
P16 & El conjunto de soluciones X\_f es el conjunto de las soluciones factibles del problema de programación lineal si todas las x\_f son factibles & Verdadero \\
\hline
P17 & La diferencia entre una solución óptima y una subóptima se debe a la forma de la función objetivo & Falso \\
\hline
P18 & Dadas las restricciones de no negatividad de un problema lineal de maximización, si la función objetivo resulta negativa, la solución es infactible & Falso \\
\hline
P19 & El precio sombra es el incremento del valor de la función objetivo al agregar variables de holgura o exceso & Falso \\
\hline
P20 & Una solución x⋆ de un NLP se define como un máximo local si ∀x∈Vx⋆∖x⋆, f(x⋆)>f(x) & Falso \\
\hline
P21 & El precio sombra es la sensibilidad a cambios en los lados derechos de las restricciones & Verdadero \\
\hline
P22 & En el algoritmo de colonia de abejas la función fitness para valores positivos de la función objetivo es una distribución de probabilidad & Falso \\
\hline
P23 & Al resolver un problema de programación lineal por fuerza bruta se debe guardar el máximo a cada iteración para compararlo con los demás valores & Falso \\
\hline
P24 & Al incrementar o reducir un recurso, se puede determinar que las variables básicas no cambiarán si el cambio total en la función objetivo dividido por el cambio en el recurso es igual al precio sombra & Verdadero \\
\hline
P25 & Si x⋆ es un mínimo local y f(x) es continuamente diferenciable en Vx⋆, entonces ∇f(x⋆)=0 & Verdadero \\
\hline
P26 & Un precio sombra negativo en un tablero final del método simplex implica que al hacer crecer el lado derecho de la restricción, el valor de la función objetivo decrece & Falso \\
\hline
P27 & En un problema generalizado de optimización se busca optimizar cualquier función de las variables de decisión establecida & Verdadero \\
\hline
P28 & Las variables de holgura se agregan cuando hay restricciones de tipo "menor o igual que" & Verdadero \\
\hline
P29 & En el método simplex, sacar una variable de decisión básica para introducir una variable de holgura/exceso siempre disminuye el valor de la función objetivo & Falso \\
\hline
P30 & La vecindad Vx⋆ se define como el conjunto de puntos en torno al óptimo de una función & Verdadero \\
\hline
P31 & Para resolver un problema de programación no lineal mediante multiplicadores de Lagrange debe resolverse un sistema con el gradiente de la función de Lagrange igualado a cero & Falso \\
\hline
P32 & El precio sombra es la sensibilidad del modelo a cambios en los parámetros de la función objetivo & Falso \\
\hline
P33 & Las variables de holgura se agregan en la función objetivo durante la primera fase & Falso \\
\hline
P34 & En un modelo de programación lineal, la función objetivo es una recta o plano que tiene el mismo valor para cualquier punto en el que se evalúe & Verdadero \\
\hline
P35 & La fase de reproducción de un algoritmo genético utilizado para maximizar un problema toma en consideración los individuos con menos mutaciones & Falso \\
\hline
P36 & El precio sombra es el incremento del valor de la función objetivo al agregar variables de holgura o exceso & Falso \\
\hline
P37 & La diferencia principal entre el gradiente descendente y el método de Newton es la utilización de la matriz hessiana para definir la dirección de búsqueda & Verdadero \\
\hline
P38 & Los algoritmos genéticos paralelos de grano fino no permiten el cruce entre individuos de vecindarios adyacentes & Falso \\
\hline
P39 & Para múltiples variables, el método de Lagrange debe calcular el gradiente de la función objetivo y de cada restricción por separado & Verdadero \\
\hline
P40 & El criterio de la hessiana 2×2 H(x)=f₁₁·f₂₂−f₁₂² solo puede usarse si la hessiana es simétrica & Verdadero \\
\hline
P41 & Cambiar una variable de decisión básica por una de holgura/exceso siempre reduce la función objetivo & Falso \\
\hline
P42 & Un precio sombra positivo siempre implica que aumentar el recurso incrementa la función objetivo & Falso \\
\hline
P43 & Si ∇²f(x⋆) es definida negativa y ∇f(x⋆)=0, entonces x⋆ es un mínimo local estricto & Falso \\
\hline
P44 & En un problema generalizado de optimización se optimiza cualquier función de las variables de decisión & Verdadero \\
\hline
P45 & El algoritmo ramificar y acotar establece cotas superiores e inferiores & Verdadero \\
\hline
P46 & El método simplex no es un algoritmo codicioso & Falso \\
\hline
P47 & El precio sombra es la derivada de la función objetivo respecto de una restricción & Verdadero \\
\hline
P48 & El precio sombra representa la utilidad marginal de un recurso & Verdadero \\
\hline
P49 & Un cruzamiento en algoritmo genético siempre genera dos hijos & Verdadero \\
\hline
P50 & El criterio H(x) no informa del tipo de punto crítico si la función es simétrica en el mismo & Verdadero \\
\hline
P51 & Una matriz es semidefinida negativa aunque alguno de sus valores propios sea mayor a cero & Falso \\
\hline
P52 & Una solución x⋆ es máximo local si f(x⋆)>f(x) ∀x∈Vx⋆∖\{x⋆\} & Falso \\
\hline
P53 & Si ΔFO/Δrecurso = precio sombra, las variables básicas no cambian & Verdadero \\
\hline
P54 & Las variables de holgura se agregan cuando hay restricciones "mayor o igual que" & Falso \\
\hline
P55 & La diferencia entre gradiente descendente y Barzilai–Borwein es solo la definición inicial de Δx & Falso \\
\hline
P56 & Con Lagrange se debe resolver un sistema con la hessiana igualada a cero & Falso \\
\hline
P57 & Las variables artificiales son básicas iniciales en restricciones "≥" en la primera fase del método de dos fases & Verdadero \\
\hline
P58 & La fila pivote se elige con el menor RHS/coeficiente (todos los coeficientes) de la columna pivote & Falso \\
\hline
P59 & En el algoritmo de colonia de abejas, las abejas espectadoras eligen la fuente probabilísticamente & Verdadero \\
\hline
P60 & Si la hessiana se anula, las condiciones suficientes de segundo orden no permiten clasificar el punto & Falso \\
\hline
P61 & Los GA paralelos de grano fino permiten cruzar individuos de vecindarios no adyacentes & Falso \\
\hline
P62 & En recocido simulado se pueden aceptar soluciones de mayor energía para escapar de mínimos locales & Verdadero \\
\hline
P63 & Resolver la versión sin integridad de un problema entero se llama relajación continua & Verdadero \\
\hline
P64 & En ramificar y acotar es fácil estimar la RAM necesaria & Falso \\
\hline
P65 & Si una función es convexa, cualquier óptimo local es global & Verdadero \\
\hline
P66 & Para un mínimo local con f diferenciable, necesariamente ∇f(x⋆)≠0 & Falso \\
\hline
P67 & Barzilai–Borwein ajusta t para acelerar la convergencia & Verdadero \\
\hline
P68 & El método de dos fases se usa cuando hay restricciones "≥" & Falso \\
\hline
P69 & Una matriz es semidefinida positiva si y solo si todos sus eigenvalores son ≥0 & Verdadero \\
\hline
P70 & En recocido simulado la temperatura controla la duración del algoritmo & Falso \\
\hline
P71 & La derivada numérica siempre mejora al reducir Δx & Falso \\
\hline
P72 & Cada Δx\_i debe cambiar solo si x\_i cambia & Verdadero \\
\hline
P73 & Una función es convexa si f(λxa+(1−λ)xb) < λf(xa)+(1−λ)f(xb) para todo λ & Falso \\
\hline
P74 & El conjunto X\_f es factible si todas las x\_f lo son & Verdadero \\
\hline
P75 & La diferencia entre óptimo y subóptimo se debe a la forma de la función objetivo & Falso \\
\hline
P76 & Si la función objetivo resulta negativa con variables no negativas, la solución es infactible & Falso \\
\hline
P77 & La fitness en colonia de abejas es una distribución de probabilidad & Falso \\
\hline
P78 & En fuerza bruta se guarda el máximo en cada iteración & Falso \\
\hline
P79 & Si ΔFO/Δrecurso = precio sombra, la función objetivo no cambia & Falso \\
\hline
P80 & Para un mínimo local con f diferenciable se cumple ∇f(x⋆)=0 & Verdadero \\
\hline
P81 & Un precio sombra negativo puede aparecer en el tablero final del simplex & Falso \\
\hline
P82 & La vecindad Vx⋆ es el conjunto de puntos a distancia ≤ε de x⋆ & Verdadero \\
\hline
P83 & La función objetivo es el producto interno C·X & Verdadero \\
\hline
P84 & El precio sombra es la derivada de una restricción saturada respecto a la FO & Falso \\
\hline
P85 & El precio sombra cambia al modificar coeficientes de la función objetivo & Falso \\
\hline
P86 & La validación es el proceso de prueba y depuración del modelo & Verdadero \\
\hline
P87 & La función objetivo es la combinación lineal de los recursos & Falso \\
\hline
P88 & En la segunda fase del método de dos fases la FO final siempre es cero & Falso \\
\hline
P89 & El precio sombra indica la utilidad de cambiar un recurso, no su costo & Verdadero \\
\hline
P90 & Al pivotear se obtiene el nuevo valor básico y su aporte marginal a la FO & Verdadero \\
\hline
P91 & RHS negativo viola las restricciones de no negatividad & Verdadero \\
\hline
P92 & El simplex termina cuando no hay coeficientes positivos en la FO & Falso \\
\hline
P93 & x⋆ es máximo global si f(x⋆) ≥ f(x) para todo x factible & Verdadero \\
\hline
P94 & Si un punto es mínimo local y la hessiana es continua, la hessiana allí será semidefinida negativa & Falso \\
\hline
P95 & Una heurística sacrifica precisión por velocidad & Verdadero \\
\hline
P96 & El teorema de Taylor aproxima funciones con derivadas sucesivas & Verdadero \\
\hline
P97 & Con coeficientes de FO negativos, el método gráfico da la solución más alejada del origen & Falso \\
\hline
P98 & La diferencia entre información determinística y probabilística es la presencia del azar & Verdadero \\
\hline
P99 & En la aproximación de Taylor la precisión depende sobre todo de Δx & Verdadero \\
\hline
P100 & Los multiplicadores de Lagrange son costos reducidos de la FO & Falso \\
\hline
P101 & La derivada numérica calcula la pendiente de la secante, no de la tangente exacta & Verdadero \\
\hline
P102 & Una función es cóncava si f(λxa+(1−λ)xb) ≤ λf(xa)+(1−λ)f(xb) & Falso \\
\hline
P103 & Las condiciones suficientes de segundo orden permiten decidir si un punto es mínimo o máximo local estricto & Verdadero \\
\hline
P104 & El método de dos fases se aplica cuando hay restricciones de tipo 'mayor o igual que'. & Verdadero \\
\hline
P105 & En un modelo de programación lineal, la función objetivo es el producto vectorial del vector de variables de decisión y el vector de coeficientes. & Verdadero \\
\hline
P106 & Un precio sombra positivo en un tablero final del método simplex implica que al hacer crecer el lado derecho de la restricción, el valor de la función objetivo siempre crecerá. & Falso \\
\hline
P107 & El precio sombra es la sensibilidad a cambios en los lados derechos de las restricciones. & Verdadero \\
\hline
P108 & El precio sombra es la derivada de una restricción saturada con respecto a la función objetivo. & Falso \\
\hline
P109 & Supongamos que el coeficiente más negativo de la primera fila en una iteración del método simplex para un problema de maximización corresponde a una variable de holgura o de exceso, y la fila pivote corresponde a una variable de decisión. Por tanto, el debería sacarse del grupo de variables básicas a la variable de decisión e introducir al mismo la respectiva variable de holgura/exceso. No obstante, no debe hacerse este cambio pues la función objetivo disminuiría en valor. & Falso \\
\hline
P110 & Al incrementar/reducir un recurso, se puede asegurar que el valor de la función objetivo no cambiará si el cambio total en la función objetivo dividido por el cambio en el recurso es igual al precio sombra. & Falso \\
\hline
P111 & La fila pivote se selecciona a partir del menor valor que se obtiene al dividir el lado derecho de las restricciones por los coeficientes de la columna pivote. & Falso \\
\hline
P112 & Las variables de holgura se agregan cuando hay restricciones de tipo 'mayor o igual que'. & Falso \\
\hline
P113 & El precio sombra es la sensibilidad del modelo a cambios en los parámetros de la función objetivo. & Falso \\
\hline
P114 & El precio sombra es el incremento del valor de la función objetivo al modificar los coeficientes de la misma. & Falso \\
\hline
P115 & La validación es el proceso de prueba y depuración del modelo matemático. & Verdadero \\
\hline
P116 & En un modelo de programación lineal, la función objetivo es la combinación lineal de los recursos. & Falso \\
\hline
P117 & El método de dos fases obligatoriamente tiene cero como valor final de la función objetivo en la segunda fase. & Falso \\
\hline
P118 & Las variables de holgura se agregan cuando hay restricciones de tipo 'menor o igual que'. & Verdadero \\
\hline
P119 & La interpretación del precio sombra impone al 'propietario' del problema la cuestión sobre el incremento/reducción de uno de los recursos, pero no expone el costo de realizar este incremento. & Verdadero \\
\hline
P120 & El precio sombra es la derivada de la función objetivo con respecto a una restricción. & Verdadero \\
\hline
P121 & Al pivotear sobre una columna pivote, lo que estamos es determinando el valor de la nueva variable básica y su aporte marginal a la función objetivo. & Verdadero \\
\hline
P122 & Ningún coeficiente del lado derecho de las restricciones puede ser negativo porque esto violaría las restricciones de no negatividad del modelo. & Verdadero \\
\hline
P123 & En un modelo de programación lineal, la función objetivo es una recta o plano que tiene el mismo valor para cualquier punto en el que se evalúe. & Verdadero \\
\hline
P124 & La diferencia entre el método básico del gradiente descendente y el de actualización de t por el método de Barzilai-Borwein radica en la definición inicial del paso de salto Δx. & Falso \\
\hline
P125 & Si una función es convexa, cualquier mínimo o máximo local es también un mínimo o máximo global. & Verdadero \\
\hline
P126 & El proceso de estudio de la solución obtenida, evaluando la flexibilidad de la misma con respecto a los parámetros suministrados al modelo matemático se denomina validación de la solución. & Falso \\
\hline
P127 & Para que un punto x⋆ sea considerado como un máximo global, no debe cumplirse que f(x⋆)≥f(x),∀x∈Rn. & Falso \\
\hline
P128 & Dadas las restricciones de no negatividad de un problema lineal de maximización, si la función objetivo tiene como resultado un valor negativo, esta solución es infactible. & Falso \\
\hline
P129 & Al resolver un problema de programación no lineal por el método de fuerza bruta, la complejidad del problema depende del número de volúmenes en que se subdivide la región factible. & Verdadero \\
\hline
P130 & El precio sombra es el incremento del valor de la función objetivo al agregar variables de holgura o exceso. & Falso \\
\hline
P131 & Suponga que ∇²f(x) es continua en Vx⋆, que ∇f(x⋆)=0 y que ∇²f(x⋆) es definida negativa, entonces x⋆ es un mínimo local estricto de f(x⋆). & Falso \\
\hline
P132 & Una función puede ser cóncava y convexa en Vx⋆, si x⋆ es un punto crítico. & Verdadero \\
\hline
P133 & Debido a las condiciones suficientes de segundo orden si la matriz hessiana se anula en el punto evaluado no es posible determinar si es un mínimo o máximo local estricto. & Falso \\
\hline
P134 & Las variables artificiales se agregan para poder tener coeficientes negativos en la función objetivo en las restricciones de tipo 'mayor o igual que' en la primera fase del modelo de dos fases. & Verdadero \\
\hline
P135 & En un problema generalizado de optimización se busca optimizar cualquier función de las variables de decisión establecida. & Verdadero \\
\hline
P136 & El método de dos fases se aplica únicamente en problemas de minimización. & Falso \\
\hline
P137 & El criterio H(x)=∂²f(x)/∂x₁²⋅∂²f(x)/∂x₂²−(∂²f(x)/∂x₁∂x₂)² para matrices hessianas 2x2 solo puede utilizarse si estas matrices son simétricas. & Verdadero \\
\hline
P138 & Una heurística sacrifica precisión por velocidad. & Verdadero \\
\hline
P139 & Un precio sombra negativo en un tablero final del método simplex implica que al hacer crecer el lado derecho de la restricción, el valor de la función objetivo decrece. & Falso \\
\hline
P140 & El teorema de Taylor es una aproximación de una función a partir de sus derivadas. & Verdadero \\
\hline
P141 & Al resolver un problema de maximización de programación lineal por el método gráfico, siendo negativos todos los coeficientes de la función objetivo, la solución se encontrará en el punto más alejado al origen del sistema de coordenadas. & Falso \\
\hline
P142 & La diferencia entre información determinística y probabilistica es la presencia del azar. & Verdadero \\
\hline
P143 & La aproximación de la función estudiada por el teorema de Taylor se realiza en Vx, por lo que el principal factor en la precisión de la aproximación será Δx. & Verdadero \\
\hline
P144 & Los multiplicadores de Lagrange se interpretan como los costos reducidos de la función objetivo. & Falso \\
\hline
P145 & La derivada explicita se define como la pendiente de la tangente a f(x) en x. La definición de derivada numérica usada en el curso no corresponde exactamente a esta definición, pues en realidad busca el valor de la pendiente de la secante a f(x) en x. & Verdadero \\
\hline
P146 & Se dice que una función f(x) es concava si, para cada xa y xb, y para cada λ∈[0,1], f(λxa+(1−λ)xb)≤λf(xa)+(1−λ)f(xb). & Falso \\
\hline
P147 & La derivada numérica de una función es más precisa mientras más pequeño sea el valor de Δx, sin importar que tan pequeño sea el mismo. & Falso \\
\hline
P148 & Las condiciones suficientes de segundo orden permiten determinar cuando un punto es un mínimo o máximo local estricto de una función. & Verdadero \\
\hline
P149 & En la forma definida en clases, el método simplex itera hasta que no encuentra valores positivos en los coeficientes de la función objetivo. & Falso \\
\hline
P150 & Una solución x⋆ de un problema de programación no lineal se define como un máximo global si f(x⋆)≥f(xf),∀x∈Rn. & Verdadero \\
\hline
P151 & Las condiciones necesarias de segundo orden establecen que si un punto evaluado es un mínimo y la matriz hessiana existe y es continua en la vecindad de ese punto, entonces la matriz hessiana evaluada en ese punto será semidefinida negativa. & Falso \\
\hline
P152 & Para resolver un problema de programación no lineal mediante el método de los multiplicadores de Lagrange debe resolverse un sistema de ecuaciones con la matriz hessiana igualada a cero. & Falso \\
\hline
P153 & La diferencia principal entre el algoritmo del gradiente descendente y el método de Newton es la utilización de la matriz hessiana en este último para determinar la dirección de búsqueda del punto óptimo. & Verdadero \\
\hline
P154 & Las variables artificiales son las variables básicas de inicio en las restricciones de tipo 'mayor o igual que' en la primera fase del método de dos fases. & Verdadero \\
\hline
P155 & La fase de reproducción de un algoritmo genético utilizado para maximizar un problema toma en consideración los individuos con menos mutaciones para identificar nuevas soluciones. & Falso \\
\hline
P156 & Una matriz es indefinida cuando alguno de sus valores propios sea igual a cero. & Falso \\
\hline
P157 & En un algoritmo de colonias de hormigas cada hormiga toma una decisión probabilística proporcional únicamente a las feromonas acumuladas para cada opción. & Falso \\
\hline
P158 & Los cromosomas para los algoritmos genéticos corresponden a un valor de un gen en un individuo. & Falso \\
\hline
P159 & En un algoritmo genético, los cromosomas con mayor aptitud son siempre los padres de la nueva generación. & Falso \\
\hline
P160 & En un problema de programación no lineal las soluciones se encuentran en los vértices de la región factible. & Falso \\
\hline
P161 & En el algoritmo de colonia de abejas las abejas espectadoras determinan probabilísticamente la fuente a explorar. & Verdadero \\
\hline
P162 & Al resolver un problema de maximización de programación lineal por el método gráfico, siendo positivos todos los coeficientes de la función objetivo y teniendo dos restricciones de tipo mayor o igual que, el problema es factible. & Verdadero \\
\hline
P163 & Para poder aplicar el método de los multiplicadores de Lagrange es necesario convertir las desigualdades de las restricciones en igualdades, agregando variables de holgura o exceso. & Verdadero \\
\hline
P164 & Al pivotar sobre una columna pivote, lo que estamos es determinando el valor de la nueva variable básica y su aporte marginal a la función objetivo. & Verdadero \\
\hline
P165 & Una matriz es semidefinida positiva si y sólo si todos su valores propios son mayores o iguales a cero. & Verdadero \\
\hline
P166 & Al resolver un problema de programación no lineal por el método de fuerza bruta, la complejidad del problema depende del número de volúmenes en que se subdivide la región factible. & Verdadero \\
\hline
P167 & Las condiciones necesarias de segundo orden se usan para determinar cuando el gradiente se anula en el óptimo y, a la vez, cuando la matriz hessiana es semidefinida positiva o negativa (dependiendo de si es un mínimo o un máximo). & Verdadero \\
\hline
P168 & La actualización de t en el método de Barzilai-Borwein tiene como objetivo acelerar la convergencia a un punto crítico local. & Verdadero \\
\hline
P169 & El teorema de Taylor es una aproximación de una función a partir de sus derivadas. & Verdadero \\
\hline
P170 & El método de dos fases obligatoriamente tiene cero como valor final de la función objetivo en la segunda fase. & Falso \\
\hline
P171 & Al incrementar/reducir un recurso, se puede asegurar que el valor de la función objetivo no cambiará si el cambio total en la función objetivo dividido por el cambio en el recurso es igual al precio sombra. & Falso \\
\hline
P172 & El precio sombra es el incremento del valor de la función objetivo al modificar los coeficientes de la misma. & Falso \\
\hline
P173 & El criterio para matrices hessianas 2x2 solo puede utilizarse si estas matrices son simétricas. & Verdadero \\
\hline
P174 & La derivada numérica de una función es más precisa mientras más pequeño sea el valor de Δx, sin importar que tan pequeño sea el mismo. & Falso \\
\hline
P175 & El método simplex no es un algoritmo greedy (codicioso). & Falso \\
\hline
P176 & Las variables de holgura se agregan cuando hay restricciones de tipo 'menor o igual que'. & Verdadero \\
\hline
P177 & Para probar que una función tiene un óptimo global solo es necesario probar que ésta es convexa. & Falso \\
\hline
P178 & Ningún coeficiente del lado derecho de las restricciones puede ser negativo porque esto violaría las restricciones de no negatividad del modelo. & Verdadero \\
\hline
P179 & Las condiciones suficientes de segundo orden permiten determinar cuando un punto es un mínimo o máximo local estricto de una función. & Verdadero \\
\hline
P180 & Una matriz es semidefinida negativa aunque alguno de sus valores propios sea mayor a cero. & Falso \\
\hline
P181 & El precio sombra es la derivada de la función objetivo con respecto a una restricción. & Verdadero \\
\hline
P182 & La vecindad Vx* se define como el conjunto de puntos en torno al óptimo de una función. & Verdadero \\
\hline
P183 & Las variables de holgura se agregan en la función objetivo durante la primera fase. & Falso \\
\hline
P184 & El precio sombra solo se puede observar en restricciones saturadas. & Verdadero \\
\hline
P185 & En un modelo de programación lineal, la función objetivo es la combinación lineal de los recursos. & Falso \\
\hline
P186 & Dadas las restricciones de no negatividad de un problema lineal de maximización, si la función objetivo tiene como resultado un valor negativo, esta solución es infactible. & Falso \\
\hline
P187 & Para resolver un problema de programación no lineal mediante el método de los multiplicadores de Lagrange debe resolverse un sistema de ecuaciones con la matriz hessiana igualada a cero. & Falso \\
\hline
P188 & En un problema de programación no lineal las soluciones se encuentran en los vértices de la región factible. & Falso \\
\hline
P189 & La fila pivote se selecciona a partir del menor valor que se obtiene al dividir el lado derecho de las restricciones por los coeficientes de la columna pivote. & Falso \\
\hline
P190 & Si x⋆ es un mínimo local y f(x) es continuamente diferenciable en Vx⋆, entonces ∇f(x⋆)=0. & Verdadero \\
\hline
P191 & Al resolver un problema de programación lineal por fuerza bruta se debe guardar el máximo a cada iteración para hacer compararlo con los valores de los demás volúmenes. & Falso \\
\hline
P192 & La validación es el proceso de prueba y depuración del modelo matemático. & Verdadero \\
\hline
P193 & El precio sombra es la utilidad marginal de un recurso. & Verdadero \\
\hline
P194 & En un modelo de programación lineal, la función objetivo es una recta o plano que tiene el mismo valor para cualquier punto en el que se evalúe. & Verdadero \\
\hline
P195 & Una función puede ser cóncava y convexa en V\_x\^{}⋆, si x\^{}⋆ es un punto crítico. & Verdadero \\
\hline
P196 & Las variables artificiales se agregan para poder tener coeficientes negativos en la función objetivo en las restricciones de tipo 'mayor o igual que' en la primera fase del modelo de dos fases. & Verdadero \\
\hline
P197 & Una solución x\^{}⋆ de un problema de programación no lineal se define como un máximo global si f(x\^{}⋆)≥f(x) para todo x factible. & Verdadero \\
\hline
P198 & El método simplex garantiza que la solución óptima se encuentra en un vértice de la región factible. & Verdadero \\
\hline
P199 & Una matriz es semidefinida negativa si y sólo si xᵀMx≤0, ∀x∈ℝⁿ∖\{0\}. & Verdadero \\
\hline
P200 & La diferencia entre información determinística y probabilística es la presencia del azar. & Verdadero \\
\hline
P201 & La derivada explícita se define como la pendiente de la tangente a f(x) en x. La definición de derivada numérica usada en el curso no corresponde exactamente a esta definición, pues en realidad busca el valor de la pendiente de la secante a f(x) en x. & Verdadero \\
\hline
P202 & El método de dos fases se aplica cuando no hay restricciones de tipo 'mayor o igual que'. & Falso \\
\hline
P203 & La aproximación de la función estudiada por el teorema de Taylor se realiza en V\_x, por lo que el principal factor en la precisión de la aproximación será Δx. & Verdadero \\
\hline
P204 & Se dice que una función f(x) es cóncava si, para cada x\_a y x\_b, y para cada λ∈[0,1], f(λx\_a+(1−λ)x\_b)≤λf(x\_a)+(1−λ)f(x\_b). & Falso \\
\hline
P205 & La matriz hessiana puede ser usada para determinar el tipo de punto crítico encontrado. & Verdadero \\
\hline
P206 & Las variables artificiales son las variables básicas de inicio en las restricciones de tipo 'mayor o igual que' en la primera fase del método de dos fases. & Verdadero \\
\hline
P207 & La matriz hessiana es una matriz de derivada parciales de segundo orden de una función continuamente diferenciable dos veces con respecto a pares de variables. Por tanto, sabemos que es la matriz hessiana es una matriz simétrica. & Verdadero \\
\hline
P208 & La interpretación del precio sombra impone al 'propietario' del problema la cuestión sobre el incremento/reducción de uno de los recursos, pero no expone el costo de realizar este incremento. & Verdadero \\
\hline
P209 & Los multiplicadores de Lagrange se interpretan como los costos reducidos de la función objetivo. & Falso \\
\hline
P210 & Se dice que una función f(x) es convexa si, para cada x\_a y x\_b, y para cada λ∈[0,1], f(λx\_a+(1−λ)x\_b)<λf(x\_a)+(1−λ)f(x\_b). & Falso \\
\hline
P211 & Para el caso de múltiples variables, el método de los multiplicadores de Lagrange debe calcular tanto el gradiente de la función objetivo como el gradiente de cada restricción por separado. & Verdadero \\
\hline
P212 & Si x⋆ es un mínimo local y f(x) es continuamente diferenciable en Vx⋆, entonces ∇f(x⋆)≠0. & Falso \\
\hline
P213 & En un modelo de programación lineal, la función objetivo es el producto vectorial del vector de variables de decisión y el vector de coeficientes. & Verdadero \\
\hline
P214 & El criterio H(x)=∂²f(x)/∂x₁²⋅∂²f(x)/∂x₂²−(∂²f(x)/∂x₁∂x₂)² para matrices hessianas 2x2 es incapaz de dar información sobre un punto crítico si la función es simétrica con respecto a las variables en el mismo. & Verdadero \\
\hline
P215 & Al resolver un problema de minimización de programación lineal por el método gráfico, buscamos el vértice de la región factible más cercano al origen del sistema de coordenadas. & Falso \\
\hline
P216 & Cuando en el método de Newton se evalúa ∥∇x\_\{n+1\}∥<∥∇x\_n∥, se establece, implícitamente, que lo que se busca es un máximo. & Falso \\
\hline
P217 & La diferencia entre una solución óptima y una subóptima se debe a la forma de la función objetivo. & Falso \\
\hline
P218 & Dado que el gradiente de una función es el vector de derivada parciales de la función con respecto a cada una de las variables, y el problema de redondeo al calcular la derivada numérica, es necesario que cada valor del vector delta x sea diferente únicamente cuando las componentes del vector x son diferentes. & Verdadero \\
\hline
P219 & Si una función es convexa, cualquier mínimo o máximo local es también un mínimo o máximo global. & Verdadero \\
\hline
P220 & En el siguiente tablero final de un modelo resuelto por el método simplex, las holgura menos rentable es la correspondiente a la restricción 2. & Falso \\
\hline

\end{longtable}

\end{document}
